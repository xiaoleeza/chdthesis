% !TeX root = ../main.tex

\chapter{图表及其附注}
图和表应安排在主体部分中第1次提及该图、表的文字下方。
当图或表不能安排在该页时,应安排在该页的下一页。

\section{表格}

编制表格应简单明了,表达一致,明晰易懂,表文呼应、内容一致。
排版时表格字号略小,或变换字体,尽量不分页,尽量不跨节。
表格太大需要转页是,需要在续表上方注明“续表”,表头页应重复排出。

\subsection{三线表}

一律使用三线表,与文字齐宽,上下边线,线粗1.5 磅,表内线,线粗1 磅,
可直接使用\verb|\toprule|、\verb|\midrule|、\verb|\cmidrule{}|、\verb|\bottomrule|生成表格线,
如表~\ref{tab:exampletable1},使用\verb|tabular*|环境指定表格宽度,
并利用\verb|@{\extracolsep{\fill}}|使内容自动伸展。
\begin{table}[htb]
  \centering
  \caption{表号和表题在表的正上方}
  \label{tab:exampletable1}
  \begin{tabular*}{0.6\linewidth}{c@{\extracolsep{\fill}}*{1}{c}}
    \toprule
    类型   & 描述                                       \\
    \midrule
    挂线表 & 挂线表也称系统表、组织表,用于表现系统结构 \\
    无线表 & 无线表一般用于设备配置单、技术参数列表等   \\
    卡线表 & 卡线表有完全表,不完全表和三线表三种       \\
    \bottomrule
  \end{tabular*}
\end{table}

\subsection{附注}
图、表中若有附注时,附注各项的序号一律用“附注 + 阿拉伯数字 + 冒号” ,如:“附注1:”。
附注写在图、表的下方,一般采用5号宋体。

针对撰写规范的规定,模板自定义了\verb|notetabular|环境创建带附注的表格,如表~\ref{tab:exampletable2}。
使用方法与\verb|tabular*|类似,只是多了\verb|\tnote|参数。
\begin{table}
  \centering
  \caption{创建带附注表格}
  \label{tab:exampletable2}
  \begin{notetabular}{0.6\linewidth}{c@{\extracolsep{\fill}}*{2}{c}}  % 基本参数与tabular*一致
    {
      \tnote{竖向合并单元格,可能需要手动调整内容位置}
      \tnote[\hspace{2\ccwd}附注4]{自定义编号和缩进}
    }  % 附注内容须提前输入
    \toprule
    \multirow{2}{*}[-1.5mm]{Header 1\tmark} & \multicolumn{2}{c}{Header 2}\\
    \cmidrule{2-3}
    & Header 2.1& Header 2.2\\
    \midrule
    foo 1\tmark[4] & foo 2\tmark[4] & foo 3\\
    \bottomrule
  \end{notetabular}
\end{table}

\subsection{长表格}

如某个表需要转页接排,在随后的各页上应重复表的编排,
编号后跟表题(可省略)和“(续)”,
可以使用\pkg{longtable}宏包创建跨页表格。
续表应重复表头和关于单位的陈述。

\begin{longtable}{l@{\extracolsep{\fill}}*{1}{l}}
  \caption{\pkg{longtable}命令汇总\label{tab:exampletable3}} \\
  \toprule
  命令& 说明\\
  \midrule
  \endfirsthead
  \multicolumn{2}{l}{(续)}\\
  \toprule
  命令& 说明\\
  \midrule
  \endhead
  \bottomrule
  \endfoot
  \bottomrule
  \bottomrule[0pt]  % 添加间距
  \multicolumn{2}{l}{附注1:该命令产生的是页脚注,见\ref{section:footnote}节;}\\
  \multicolumn{2}{l}{附注2:使用footnotemark配合endlatfoot创建附注。}\\
  \endlastfoot
  不填写(默认)& 表格居中\\
  \verb|[c]|& 表格居中\\
  \verb|[l]|& 表格左对齐\\
  \verb|[r]|& 表格右对齐\\
  \verb|\\|& 结束一行表格\\
  \verb|\\[距离]|& 结束一行,并增加额外间距\\
  \multicolumn{2}{c}{(以下是为方便演示增加的额外间距)}\\[13cm]
  \verb|\\*|& 结束一行,并禁止在此分页\\*
  \verb|\kill|& 当前行不输出,只参与宽度计算\\
  xxxxxxxxxxxxxxxxxxxxxxxxxxxxxxxxxx&xxxxxxxxxxxxxxxxxxxxxxxxxxxxxxxxxxx\kill
  \verb|\endhead|& 此命令以上部分是每页的表头\\
  \verb|\endfirsthead|& 此命令以上部分是表格第一页的表头\\
  \verb|\endfoot|& 此命令以上部分是每页的表尾\\
  \verb|\endlastfoot|& 此命令以上部分是表格最后一页的表尾\\
  \verb|\caption{标题}|& 生成带编号的标题\\
  \verb|\caption*{标题}|& 生成不带编号的标题\\
  \verb|\newpage|& 强制分页\\
  \verb|\pagebreak[程度]|& 允许分页的程度(0-4)\\
  \verb|\nopagebreak[程度]|& 禁止分页的程度(0-4)\\
  \verb|\footnote|\footnotemark& 使用脚注,注意不能用在表格头尾\\
  \verb|\footnotemark|\footnotemark& 单独产生脚注编号,不能用在表格头尾\\
  \verb|\footnotetext|& 单独产生脚注文字\\
  \verb|\LTleft|& 对齐方式留空时,表格左边的间距,默认为\verb|\fill|\\
  \verb|\LTright|& 对齐方式留空时,表格右边的间距,默认为\verb|\fill|\\
  \verb|\LTpre|& 表格上方间距,默认为\verb|\bigskipamount|\\
  \verb|\LTpost|& 表格下方间距,默认为\verb|\bigskipamount|\\
  \verb|\LTcapwidth|& 表格标题的宽度,默认为4\,in\\
\end{longtable}


\section{图}
图包括曲线图、结构图、示意图、图解、框图、流程图、记录图、布置图、地图、照片、图版等。
图应具有“自明性”,即只看图、图题和图例,不阅读正文,就可理解图意。
图的编号应采用阿拉伯数字分章依续编号,如:“图3.2”。

\subsection{插入图片}
有的同学可能听说“\LaTeX{} 只能使用 eps 格式的图片”,甚至把 jpg 格式转为 eps。
事实上,这种做法已经过时。
而且每次编译时都要要调用外部工具解析 eps,导致降低编译速度。
所以我们推荐矢量图直接使用 pdf 格式,位图使用 jpeg 或 png 格式。
\begin{figure}[htb]
  \centering
  \includegraphics[width=0.3\textwidth]{chdcolorlogo.eps}
  \caption{插入eps图片}
  \label{fig:logo}
  \note{附注1:现已不推荐使用。}
\end{figure}

\pkg{Texlive}中有许多自带的工具,比如可以利用\pkg{epstopdf}将eps图片转为pdf,
利用\pkg{pdfcrop}裁剪pdf图片,这些工具可在
\,\verb|Texlive|\verb|\版本|\verb|\bin|\verb|\win32|\,文件夹中找到,
本模板\pkg{figure}文件夹中的pdf即是通过\pkg{epstopdf}得到的。
\begin{figure}[htb]
  \centering
  \includegraphics[width=0.3\textwidth]{chdcolorlogo.pdf}
  \caption{插入pdf图片}
  \label{fig:logo}
\end{figure}

关于图片的并排,推荐使用较新的 \pkg{subcaption} 宏包,
不建议使用 \pkg{subfigure} 或 \pkg{subfig} 等宏包。
