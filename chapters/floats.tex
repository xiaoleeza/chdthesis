% !TeX root = ../main.tex

\chapter{图表及其附注}
图和表应安排在主体部分中第1次提及该图、表的文字下方。
当图或表不能安排在该页时,应安排在该页的下一页。

\section{表格}

编制表格应简单明了,表达一致,明晰易懂,表文呼应、内容一致。
排版时表格字号略小,或变换字体,尽量不分页,尽量不跨节。
表格太大需要转页是,需要在续表上方注明“续表”,表头页应重复排出。

\subsection{三线表}

一律使用三线表,与文字齐宽,上下边线,线粗1.5 磅,表内线,线粗1 磅,
可直接使用\verb|\toprule|、\verb|\midrule|、\verb|\cmidrule{}|、\verb|\bottomrule|生成表格线,
如表~\ref{tab:exampletable1},使用\verb|tabular*|环境指定表格宽度,
并利用\verb|@{\extracolsep{\fill}}|使内容自动伸展。
\begin{table}[htb]
  \centering
  \caption{表号和表题在表的正上方}
  \label{tab:exampletable1}
  \begin{tabular*}{0.6\linewidth}{c@{\extracolsep{\fill}}*{1}{c}}
    \toprule
    类型   & 描述                                       \\
    \midrule
    挂线表 & 挂线表也称系统表、组织表,用于表现系统结构 \\
    无线表 & 无线表一般用于设备配置单、技术参数列表等   \\
    卡线表 & 卡线表有完全表,不完全表和三线表三种       \\
    \bottomrule
  \end{tabular*}
\end{table}

\subsection{附注}
图、表中若有附注时,附注各项的序号一律用“附注 + 阿拉伯数字 + 冒号” ,如:“附注1:”。
附注写在图、表的下方,一般采用5号宋体。

针对撰写规范的规定,模板自定义了\verb|notetabular|环境创建带附注的表格,如表~\ref{tab:exampletable2}。
使用方法与\verb|tabular*|类似,只是多了\verb|\tnote|参数。
\begin{table}
  \centering
  \caption{创建带附注表格}
  \label{tab:exampletable2}
  \begin{notetabular}{0.6\linewidth}{c@{\extracolsep{\fill}}*{2}{c}}  % 基本参数与tabular*一致
    {
      \tnote{竖向合并单元格,可能需要手动调整内容位置}
      \tnote[\hspace{2\ccwd}附注4]{自定义编号和缩进}
    }  % 附注内容须提前输入
    \toprule
    \multirow{2}{*}[-1.5mm]{Header 1\tmark} & \multicolumn{2}{c}{Header 2}\\
    \cmidrule{2-3}
    & Header 2.1& Header 2.2\\
    \midrule
    foo 1\tmark[4] & foo 2 & foo 3\\
    \bottomrule
  \end{notetabular}
\end{table}

\subsection{长表格}

\begin{longtable}{c@{\extracolsep{\fill}}*{4}{c}}
  \caption{车型一览表\label{表:车型一览表}} \\
  \toprule
  车辆类型& 车型编号& 轴数& 车型种类& 车辆图示\\
  \midrule
  \endfirsthead
  \multicolumn{5}{l}{(续表)}\\
  \toprule
  车辆类型& 车型编号& 轴数& 车型种类& 车辆图示\\
  \midrule
  \endhead
  \bottomrule
  \endfoot
  1& V1& 2& 小轿车等二轴家用车& 我是\\
  2& V3& 2& 中型双轴客车和货车& 我是\\
  3& V5& 2& 大型双轴客车和货车& 我是\\
  4& V6& 3& 三轴大型客车& 我是\\
  5& V7& 3& 三轴大型货车(双前轴)& 我是\\
  6& V8& 3& 三轴大型货车(双后轴)& 我是\\
  6& V9& 4& 四轴大型货车& 我是\\
  7& V10& 3& 三轴拖挂车& 我是\\
  8& V11& 4& 四轴拖挂车& 我是\\
  9& V12& 5& 五轴拖挂车& 我是\\
  10& V13& 5& 五轴拖挂车& 我是\\
  11& V14& 5& 五轴拖挂车& 我是\\
  12& V15& 6& 六轴拖挂车& 我是\\
  13& V16& 6& 六轴拖挂车& 我是\\
\end{longtable}


\section{插图}

有的同学可能听说“\LaTeX{} 只能使用 eps 格式的图片”,甚至把 jpg 格式转为 eps。
事实上,这种做法已经过时。
而且每次编译时都要要调用外部工具解析 eps,导致降低编译速度。
所以我们推荐矢量图直接使用 pdf 格式,位图使用 jpeg 或 png 格式。
\begin{figure}[htb]
  \centering
  \includegraphics[width=0.3\textwidth]{figures/chdcolorlogo.eps}
  \caption{图号、图题置于图的下方}
  \label{fig:logo}
  \note{注:图注的内容不宜放到图题中。}
\end{figure}

关于图片的并排,推荐使用较新的 \pkg{subcaption} 宏包,
不建议使用 \pkg{subfigure} 或 \pkg{subfig} 等宏包。
