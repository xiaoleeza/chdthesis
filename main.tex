% !TeX encoding = UTF-8
% !TeX program = xelatex
% !TeX spellcheck = en_US

\documentclass[doctor,print]{chinathesis}
% [doctor|master|bachelor] 选择学位模板,默认博士
% [chinese|english] 选择中英文模板,默认中文
% [print|pdf] 选择版式,默认为print
% [defense] 封面下方加入答辩委员会人员

\title{论文题目}
\classnumber{TP396.4} % 中图分类号,根据研究方向填写
\studentid{2019124007}%学号
\author{作者姓名} % 姓名
\professionaltype{学位类别} % 学位类别
\major{专业名称} % 专业,参考以往,专硕也只填写了专业名称
\supervisor{某某\ 教授} % 导师
\cosupervisor{某某\ 副教授}  % 副导师,不需要可注释
\submissiondate{2019年5月20日}  % 提交日期,注释后为今日
\defensedate{2019年5月30日}  % 答辩日期,注释后为今日
% \secretlevel{秘密}        % 绝密|机密|秘密,注释本行则不保密
% \secretyear{20}           % 保密年限

\entitle{Study on the Subgrade Diseases in \\Permafrost of Qinghai-Tibet}  % 排版完全按照规范,如题目超过两行可以手动调整cls中的距离
\enauthor{Wang Dazhi}
\enmajor{Information and Communication Engineering}
\ensupervisor{Prof. Zhang Dahai}
\encosupervisor{A.P.\hspace{2.5mm} Li Sanbu} % 副导师,不需要可注释
% \endate{May 1, 2017}      % Today if commented
% \enprofessionaltype{Professional degree type}
% \ensecretlevel{Secret}    % Top secret|Highly secret|Secret

% 加载宏包和配置
\usepackage{hyperref}
\usepackage{graphicx}
\graphicspath{{figures/}}
\usepackage{booktabs}
\usepackage{longtable}
\usepackage{siunitx}
\usepackage{amsthm}
\usepackage{pdfpages}
\usepackage[ruled,vlined]{algorithm2e}

% 定义斜体或正体符号,详见模板第五章
\renewcommand\vec{\symbf}
\newcommand\mat{\symbf}
\newcommand\ts{\symbfsf}
\newcommand\real{\mathbf{R}}

\begin{document}

\maketitle % 中英文封面
\makestatement % 论文独创性声明
%注释掉此两行,可以不生成中英文封面,论文独创性声明,一起由word转为pdf,由下面载入
%\includepdf[nup=1x1, delta=0mm 0mm,scale=1,pages={1-6}]{front/front.pdf}

\frontmatter
% 中英文摘要
% !TeX root = ../main.tex

\begin{abstract}
  为了提高研究生学位论文质量,统一学位论文的撰写和编辑的格式,便于信息的收集、存储、检索、利用和交流、传播。根据国家有关标准和学校实际,制定本规范。
  
学位论文的定义:学位论文是表明作者从事科学研究取得创造性的结果或有了新的见解,并以此为内容撰写而成、作为提出申请授予相应的学位时评审用的学术论文。

硕士学位论文,要求对所研究的课题有新见解或新成果,并在理论上或实践上对国民经济建设或本门学科发展具有一定的意义,表明作者在本门学科掌握坚实的基础理论和系统的专门知识,具有从事科学研究工作或独立担负专门技术工作的能力。

博士学位论文,要求对所研究的课题在科学上或专门技术上做出创造性成果,并在理论上或实践上对国民经济建设或本门学科发展具有较大的意义,表明作者在本门学科掌握坚实宽广的基础理论和系统深入的专门知识,具有独立从事科学研究工作的能力。

  本文的创新点主要有:
  \begin{itemize}
    \item 用例子来解释模板的使用方法;
    \item 用废话来填充无关紧要的部分;
    \item 一边学习摸索一边编写新代码。
  \end{itemize}

  关键词是为了文献标引工作、用以表示全文主要内容信息的单词或术语。
  关键词不超过 5 个,每个关键词中间用分号分隔。
  英文摘要的关键词与中文摘要的关键词应完全一致。

  \keywords{学位论文,\LaTeX{},模板}
\end{abstract}
\cleardoublepage
\begin{enabstract}
  An abstract of a dissertation is a summary and extraction of research work
  and contributions.
  Included in an abstract should be description of research topic and research
  objective, brief introduction to methodology and research process, and
  summarization of conclusion and contributions of the research.  An abstract
  should be characterized by independence and clarity and carry identical
  information with the dissertation.
  It should be such that the general idea and major contributions of the
  dissertation are conveyed without reading the dissertation.

  An abstract should be concise and to the point.
  It is a misunderstanding to make an abstract an outline of the dissertation
  and words ``the first chapter'', ``the second chapter'' and the like should
  be avoided in the abstract.

  Key words are terms used in a dissertation for indexing, reflecting core
  information of the dissertation.
  An abstract may contain a maximum of 5 key words, with semi-colons used in
  between to separate one another.

  \enkeywords{Dissertation, \LaTeX{}, template}
\end{enabstract}


\tableofcontents
% 图目录
\listoffigures
% 表目录
\listoftables
% 主要符号表
\input{chapters/notation}

\mainmatter
% !TeX root = ../main.tex

\chapter{绪论}
\section{论文的基本要求}
论文应立论正确、推理严谨、说明透彻、数据可靠。

论文应结构合理、层次分明、叙述准确、文字简练、文图规范。对于涉及作者创新性工作和研究特点的内容应重点论述,做到数据或实例丰富、分析全面深入。文中引用的文献资料必须注明来源,使用的计量单位、绘图规范应符合国家标准。

论文的学术水平应满足规定的要求。

学位论文主体部分的篇幅(包含图、表和公式),硕士学位论文一般为40~60页,博士学位论文一般为60~100 页。提倡文笔简洁、用语规范。
\cite{Taylor2008Medical,Pham2000Current}。
\section{论文内容}
包括:选题的背景、依据及意义;文献及相关研究综述、研究及设计方案、试验方法、装置和试验结果;理论的证明、分析和结论;重要的计算、数据、图表、曲线及相关分析;必要的附录、相关的参考文献目录等。
对于合作完成的项目,论文的内容应侧重本人的研究工作。论文中有关与指导教师或他人共同研究、试验的部分以及引用他人研究成果的部分均要明确说明。

\subsection{条标题}

\subsubsection{款标题}
\paragraph{项标题}

\subparagraph{目标题}




\section{脚注}
\label{section:footnote}

Lorem ipsum dolor sit amet, consectetur adipiscing elit, sed do eiusmod tempor
incididunt ut labore et dolore magna aliqua.
\footnote{Ut enim ad minim veniam, quis nostrud exercitation ullamco laboris
nisi ut aliquip ex ea commodo consequat.
Duis aute irure dolor in reprehenderit in voluptate velit esse cillum dolore
eu fugiat nulla pariatur.}


% !TeX root = ../main.tex

\chapter{论文的主要结构及装订顺序}
学位论文一般应由11个部分组成,装订顺序依次为:

\begin{enumerate}
\item 封面(中、英文扉页)
\item 论文独创性声明和论文知识产权权属声明
\item 中文摘要
\item 英文摘要
\item 目录
\item 图表清单及主要符号表(根据具体情况可省略)
\item 主体部分(包括绪论、正文、结论等部分)
\item 参考文献
\item 附录
\item 攻读学位期间取得的研究成果
\item 致谢
\end{enumerate}


% !TeX root = ../main.tex

\chapter{论文格式规范}

\section{论文的文字及书写}

\subsection{论文的文字}

研究生学位论文一般用中文撰写,采用国家正式公布实施的简化汉字和法定的计量单位。也可以用英文撰写,但须同时提交用中文撰写的详细摘要。
\begin{enumerate}
  \item 来华留学生学位论文的目录、主体部分和致谢等可用英文撰写;但封面、独创性声明和权属声明应用中文撰写,硕士生须同时提交3000字左右的中文详细摘要,博士生须同时提交5000字左右的中文详细摘要。
  \item 外语专业的学位论文的目录、主体部分和致谢等应用所学专业相应的语言撰写;但封面、独创性声明和权属声明应用中文撰写,摘要应使用中文和所学专业相应的语言对照撰写。
\end{enumerate}

\subsection{论文的书写}

学位论文一律采用A4(70g)幅面白色纸张,封面、封底采用白色布纹纸张,中、英文扉页、独创性声明和使用授权书采用单面印刷,从中文摘要开始采用双面印刷。

\subsection{字体和字号}

\begin{itemize}
\item 章标题:三号黑体居中
\item 节标题:四号黑体居左
\item 条标题:小四号黑体居左
\item 主体部分:小四号宋体
\item 页码:五号宋体
\item 数字和字母: Times New Roman
\end{itemize}

\section{论文页面设置}

\subsection{页边距及行距}

学位论文的上边距:25mm;下边距:25mm;左边距:30mm;右边距:20mm。
章、节、条三级标题为单倍行距,段前、段后各设为0.5行(即前后各空0.5行)。
主体部分为1.5倍行距,段前、段后无空行(即空0行)。

\subsection{页眉}

页眉的上边距为15mm,页脚的下边距为15mm。页眉内容:页眉标注从论文主体部分开始(绪论或第一章),页眉用五号宋体,居中排列。奇偶页不同。奇数页页眉为章序及章标题,例如:“第四章  路基病害类型及分布规律”,偶数页页眉为“长安大学博士学位论文”或“长安大学硕士学位论文”。格式为页眉的文字内容之下划一条横线,线长与页面齐宽。

\subsection{页码}
论文页码从“主体部分”开始,直至“致谢”结束,用五号阿拉伯数字连续编码,页码位于页脚居中。

封面(中、英文扉页)、学位论文的独创性声明和权属声明不编入页码。

摘要、目录、图表清单、主要符号表用五号小罗马数字连续编码,页码位于页脚居中。

\section{名词术语}

科技名词术语及设备、元件的名称,应采用国家标准或部颁标准中规定的术语或名称。标准中未规定的术语要采用行业通用术语或名称。全文名词术语必须统一。特殊名词或新名词应在适当位置加以说明或注解。

采用英语缩写词时,除本行业广泛应用的通用缩写词外,文中第一次出现的缩写词应该用括号注明英文全称。

\section{物理量名称、符号与计量单位}

文中所用的物理量、符号与单位一律采用国家正式公布实施的《中华人民共和国法定计量单位》及国家标准《量和单位》(GB3100~3102)。

\section{图、表及其附注}

图和表应安排在主体部分中第1次提及该图、表的文字下方。当图或表不能安排在该页时,应安排在该页的下一页。

\subsection{图}

图包括曲线图、结构图、示意图、图解、框图、流程图、记录图、布置图、地图、照片、图版等。

图应具有“自明性”,即只看图、图题和图例,不阅读正文,就可理解图意。图的编号应采用阿拉伯数字分章依续编号,如:“图3.2”。

图题应明确简短,用五号宋体加粗,数字和字母为五号Times New Roman体加粗,图的编号与图题之间应空半角2格。图的编号与图题应置于图下方的居中位置。图内文字为5号宋体,数字和字母为5号Times New Roman体。曲线图的纵横坐标必须标注“量、标准规定符号、单位”,此三者只有在不必要标明(如无量刚等)的情况下方可省略。坐标上标注的量的符号和缩略词必须与正文中一致。

照片图要求主题和主要部分的轮廓鲜明,如用放大缩小的复制品,必须清晰,反差适中。照片上应有表示目的物尺寸的标度。

\subsection{表}

一律使用三线表,与文字齐宽,上下边线,线粗1.5 磅,表内线,线粗1 磅。例如表2-1;

表的编排,一般是内容和测试项目由左至右横读,数据依序竖读。表应有自明性。

表的编号应采用阿拉伯数字分章依续编号,如:“表2.5”。表题应明确简短,用五号宋体加粗,数字和字母为五号Times New Roman体加粗,表的编号与表题之间应空半角2格。表的编号与表题应置于表上方的居中位置。表内文字为5号宋体,数字和字母为5号Times New Roman体。

如某个表需要转页接排,在随后的各页上应重复表的编排。编号后跟表题(可省略)和“(续)”,如下所示:

表2.1  路基各边界热流密度(续)

续表应重复表头和关于单位的陈述。

\subsection{附注}

图、表中若有附注时,附注各项的序号一律用“附注 + 阿拉伯数字 + 冒号” ,如:“附注1:”。附注写在图、表的下方,一般采用5号宋体。

\section{公式}

文中公式的编号采用阿拉伯数字按章编排,用圆括号括起写在右边行末,其间不加虚线。如第一章第1个公式序号为“(1.1)”, 附录A中的第1个公式为“(A1)”等。文中引用公式时,一般用“见式(1.1)”或“由公式(1.1)”。

\section{注释}

学位论文中有个别名词或情况需要解释时,可加注说明,注释用页末注(将注文放在加注页的下端),而不用篇末注(将全部注文集中在文章末尾)和行中注(夹在论文主体部分中的注)。注号用阿拉伯数字上标标注,如:“注1”

\section{保密论文}

鼓励对学位论文进行去密处理,减少不必要的保密学位论文数量。去密处理时一般应去掉应用背景,与保密项目相关的技术指标和关键数据,使论文变成纯理论和技术的研究,达到可以在论文评审人员范围内公开或阅读的程度。对于技术和方法的保密,应该通过申请专利来保护,而不是把学位论文变为保密论文。

确实需要保密的论文由指导教师根据论文的情况提出并填写《长安大学涉密学位(毕业)论文定密审批表》,
校保密工作委员会按照国家规定的保密条例进行审批。
保密审批通过的论文需在封面直接把相应的“密级$\bigstar$  % TODO: 打五角星
”及“保密期限”标注在右上角,密级按由低到高可分为“秘密”、“机密”、“绝密”三级。

% !TeX root = ../main.tex

\chapter{图表及其附注}
图和表应安排在主体部分中第1次提及该图、表的文字下方。
当图或表不能安排在该页时,应安排在该页的下一页。

\section{表格}

编制表格应简单明了,表达一致,明晰易懂,表文呼应、内容一致。
排版时表格字号略小,或变换字体,尽量不分页,尽量不跨节。
表格太大需要转页是,需要在续表上方注明“续表”,表头页应重复排出。

\subsection{三线表}

一律使用三线表,与文字齐宽,上下边线,线粗1.5 磅,表内线,线粗1 磅,
可直接使用\verb|\toprule|、\verb|\midrule|、\verb|\cmidrule{}|、\verb|\bottomrule|生成表格线,
如表~\ref{tab:exampletable1},使用\verb|tabular*|环境指定表格宽度,
并利用\verb|@{\extracolsep{\fill}}|使内容自动伸展。
\begin{table}[htb]
  \centering
  \caption{表号和表题在表的正上方}
  \label{tab:exampletable1}
  \begin{tabular*}{0.6\linewidth}{c@{\extracolsep{\fill}}*{1}{c}}
    \toprule
    类型   & 描述                                       \\
    \midrule
    挂线表 & 挂线表也称系统表、组织表,用于表现系统结构 \\
    无线表 & 无线表一般用于设备配置单、技术参数列表等   \\
    卡线表 & 卡线表有完全表,不完全表和三线表三种       \\
    \bottomrule
  \end{tabular*}
\end{table}

\subsection{附注}
图、表中若有附注时,附注各项的序号一律用“附注 + 阿拉伯数字 + 冒号” ,如:“附注1:”。
附注写在图、表的下方,一般采用5号宋体。

针对撰写规范的规定,模板自定义了\verb|notetabular|环境创建带附注的表格,如表~\ref{tab:exampletable2}。
使用方法与\verb|tabular*|类似,只是多了\verb|\tnote|参数。
\begin{table}
  \centering
  \caption{创建带附注表格}
  \label{tab:exampletable2}
  \begin{notetabular}{0.6\linewidth}{c@{\extracolsep{\fill}}*{2}{c}}  % 基本参数与tabular*一致
    {
      \tnote{竖向合并单元格,可能需要手动调整内容位置}
      \tnote[\hspace{2\ccwd}附注4]{自定义编号和缩进}
    }  % 附注内容须提前输入
    \toprule
    \multirow{2}{*}[-1.5mm]{Header 1\tmark} & \multicolumn{2}{c}{Header 2}\\
    \cmidrule{2-3}
    & Header 2.1& Header 2.2\\
    \midrule
    foo 1\tmark[4] & foo 2 & foo 3\\
    \bottomrule
  \end{notetabular}
\end{table}

\subsection{长表格}

\begin{longtable}{c@{\extracolsep{\fill}}*{4}{c}}
  \caption{车型一览表\label{表:车型一览表}} \\
  \toprule
  车辆类型& 车型编号& 轴数& 车型种类& 车辆图示\\
  \midrule
  \endfirsthead
  \multicolumn{5}{l}{(续表)}\\
  \toprule
  车辆类型& 车型编号& 轴数& 车型种类& 车辆图示\\
  \midrule
  \endhead
  \bottomrule
  \endfoot
  1& V1& 2& 小轿车等二轴家用车& 我是\\
  2& V3& 2& 中型双轴客车和货车& 我是\\
  3& V5& 2& 大型双轴客车和货车& 我是\\
  4& V6& 3& 三轴大型客车& 我是\\
  5& V7& 3& 三轴大型货车(双前轴)& 我是\\
  6& V8& 3& 三轴大型货车(双后轴)& 我是\\
  6& V9& 4& 四轴大型货车& 我是\\
  7& V10& 3& 三轴拖挂车& 我是\\
  8& V11& 4& 四轴拖挂车& 我是\\
  9& V12& 5& 五轴拖挂车& 我是\\
  10& V13& 5& 五轴拖挂车& 我是\\
  11& V14& 5& 五轴拖挂车& 我是\\
  12& V15& 6& 六轴拖挂车& 我是\\
  13& V16& 6& 六轴拖挂车& 我是\\
\end{longtable}


\section{插图}

有的同学可能听说“\LaTeX{} 只能使用 eps 格式的图片”,甚至把 jpg 格式转为 eps。
事实上,这种做法已经过时。
而且每次编译时都要要调用外部工具解析 eps,导致降低编译速度。
所以我们推荐矢量图直接使用 pdf 格式,位图使用 jpeg 或 png 格式。
\begin{figure}[htb]
  \centering
  \includegraphics[width=0.3\textwidth]{figures/chdcolorlogo.eps}
  \caption{图号、图题置于图的下方}
  \label{fig:logo}
  \note{注:图注的内容不宜放到图题中。}
\end{figure}

关于图片的并排,推荐使用较新的 \pkg{subcaption} 宏包,
不建议使用 \pkg{subfigure} 或 \pkg{subfig} 等宏包。

\input{chapters/math}
\input{chapters/citations}
% !TeX root = ../main.tex

\begin{summary}
    \section*{结论}
    通过研究得到以下结论:
    \section*{展望}
    虽然得到了一些有价值的结论,但由于篇幅和资料限制,仍有许多问题可以作进一步研究:
\end{summary}
\bibliography{bibs/refs} % 参考文献

% 附录
\appendix
\chapter{论文规范}

\backmatter
% !TeX root = ../main.tex

\cleardoublepage
\begin{publications}

对于博士学位论文,一般包括以下两项内容:

攻读博士学位期间取得的学术成果:列出攻读博士学位期间发表(含录用)的与学位论
文相关的学术论文、发明专利、著作、获奖项目等,书写格式与参考文献格式相同。

攻读博士学位期间参与的主要科研项目:列出攻读博士学位期间参与的与学位论文相关
的主要科研项目,包括项目名称,项目来源,研制时间,本人承担的主要工作。

对于硕士学位论文,列出攻读硕士学位期间发表(含录用)的与学位论文相关的学术论
文、发明专利、著作、获奖项目等,书写格式与参考文献格式相同。

\section*{学术论文}

\begin{enumerate}
\item A A A A A A A A A
\item A A A A A A A A A
\item A A A A A A A A A
\end{enumerate}

\section*{发明专利和著作}

\begin{enumerate}
\item A A A A A A A A A
\item A A A A A A A A A
\item A A A A A A A A A
\end{enumerate}

\section*{研究报告}
\begin{enumerate}
\item A A A A A A A A A
\item A A A A A A A A A
\item A A A A A A A A A
\end{enumerate}

\end{publications}

\input{chapters/acknowledgements}

\end{document}
